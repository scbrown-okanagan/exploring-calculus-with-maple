\section{Tangent Lines}
\label{sec:tangent_lines}	

\subsection*{Recommended Tutorials:}
\begin{itemize}[noitemsep]
	\item \nameref{chp:assignment_operator}, pg. \pageref{chp:assignment_operator}
	\item \nameref{chp:equation_solvers}, pg. \pageref{chp:equation_solvers}
	\item \nameref{chp:derivative}, pg. \pageref{chp:derivative}
\end{itemize}

\subsection*{Introduction:}

In this activity, students will find tangent lines to various functions, and then will display the tangent lines and function on the same graph.\\

\noindent To find the tangent line to a function $f(x)$ at $x = a$, two pieces of information are needed:
\begin{itemize}
\item The point, $(a, f(a))$.
\item The slope of the tangent line, $f'(a)$.
\end{itemize}
\marginnote{A detailed example of finding and plotting tangent lines is described on page \pageref{subsec:equation_of_tangent_line}.}
Plugging these two pieces of information into the point-slope form of a line gives the following equation.
\begin{align}
y - y_0 &= m_{tan} \cdot (x-x_0) \nonumber\\ 
y - f(a) &= f'(a) \cdot (x-a) \nonumber\\
y &= f'(a) \cdot (x-a) + f(a) \label{eq:tanline}
\end{align}\index{lines!tangent line}\index{lines!slope of a tangent line}
We will use equation \eqref{eq:tanline} for the following exercises.

\subsection*{Exercises:}
\begin{enumerate}
	\item  Consider the function $f(x) = \sqrt{9 - x}$.
    \begin{enumerate} \marginnote[-0.5cm]{Be sure to choose different names for each tangent line. If you assign a new expression to an old name, the new expression will overwrite what was previously assigned.}
    	\item Define the equations of the tangent lines at $x=0$ and $x=5$.
    	
    	\item Plot $f(x)$ and the two tangent lines on the same axes.\index{plot!multiple functions}
    	\marginnote[0.1cm]{Since you are plotting more than one
tangent line on the same axes, it is a good idea to specify plot colours.}
    \end{enumerate}
    \item Consider the function $g(x)=x{\rm e}^{x}$.\index{mathematical functions!exponential}
    \marginnote[0.3cm]{In Maple, \texttt{exp(x)} is used to denote the exponential function, ${\rm e}^x$.}\index{mathematical functions!exponential}
		\begin{enumerate}
			\item Define the equations of the tangent lines at $x=1$ and $x=-1$.
			\item Plot $g(x)$ and the two tangent lines on the same axes.
		\end{enumerate}
	\item Consider the function $h(x)=x^3-x^2-9x+9$. 
	\begin{enumerate}
		\item Find the $x$-values where the function has tangent lines with slope equal to $1$. 
		\item Find the $x$-values where the function has horizontal tangent lines\index{lines!horizontal tangent line}.
		\marginnote[-0.5cm]{To solve part (b), you need to remember what the slope of a horizontal tangent line is.}
%		\item Find the equation of the tangent line at $x=2$, then plot the function and the tangent line on the same set of axes.
	\end{enumerate}
\end{enumerate}
%\subsection*{Notes:}
%\begin{itemize}
%    \item   
%\end{itemize}