\section{Probability}\index{probability!}
\label{sec:probability}	

\subsection*{Recommended Tutorials:}
\begin{itemize}[noitemsep]
    \item \nameref{chp:assignment_operator}, pg. \pageref{chp:assignment_operator}
	\item \nameref{chp:definite_and_indefinite_Integrals}, pg. \pageref{chp:definite_and_indefinite_Integrals}
\end{itemize}

\subsection*{Introduction:}
In this activity, we define probability density functions (pdf) for some continuous probability distributions. Let’s define our probability function as $f(x)$. If $f(x)$ is a valid probability function, then
\[ \int_{-\infty}^{\infty} f(x)dx=1, \]
and $ 0  \leq f(x) \leq 1$. 
 \index{integral!improper}
 
If we want to compute the probability that a given observation is less than $a$, then we calculate
\[ P(x<a)=\int_{-\infty}^{a} f(x)dx. \]
If we want to compute the probability that the given observation is more than $a$, then we calculate
\[ P(x>a)=\int_{a}^{\infty} f(x)dx. \]
If we want to compute the probability that the given observation is between two values $a$ and $b$, then we calculate 
\[ P(a<x<b)=\int_{a}^{b} f(x)dx. \]

\begin{marginfigure}[-0.5cm]
    \hspace{-4cm}
    \centering
    \begin{tikzpicture}
    \footnotesize
    \begin{axis}[
    width=6cm,
    height=5cm,
    axis lines=middle,
    xlabel={$x$},
    ylabel={$f(x)$},
    xlabel style={below right},
    ylabel style={above left},
    xmin=0, xmax=12, xtick={0,5}, xticklabels={,$\lambda$},
    ymin=0, ymax=0.22, ytick={0,0.2}, yticklabels={,$\frac{1}{\lambda}$}
    ]
        \addplot [domain=0:12, samples=100] {1/5*exp(-1/5*x)};
    \end{axis}
    \end{tikzpicture}
    \caption{The exponential probability density function.}
    \label{fig:exponential_dist}
\end{marginfigure}
    
The exponential distribution is defined as 
\begin{equation}
    \label{eq:exponential_dist}
    f(x) = 
    \begin{cases}
        \lambda{\rm e}^{-\lambda x} & \text{if } x \geq 0 \\
        0 & \text{if } x<0,
    \end{cases}
\end{equation}
    \index{probability!exponential density function}
where $\frac{1}{\lambda}$ is the mean and standard deviation of the probability distribution.

\begin{marginfigure}
    \hspace{-4cm}
    \centering
    \begin{tikzpicture}
    \footnotesize
    \begin{axis}[
    width=6cm,
    height=5cm,
    axis lines=middle,
    xlabel={$x$},
    ylabel={$f(x)$},
    xlabel style={below right},
    ylabel style={above left},
    xmin=0, xmax=12, xtick={0,2,4,6,8,10}, xticklabels={,,,$\mu$,,},
    ymin=0, ymax=0.3, ytick={0,5}, yticklabels={,$\lambda$}
    ]
        \addplot [domain=0:12, samples=100] {1/(2*sqrt(2*pi))*exp(-(x-6)^2/(2*2^2))};
    \end{axis}
    \end{tikzpicture}
    \caption{The normal probability density function.}
    \label{fig:normal_dist}
\end{marginfigure}
    \index{probability!normal probability density function}

Define the normal distribution as 
\begin{equation}
    \label{eq:normal_dist}
    f(x) = \frac{1}{\sigma \sqrt{2\pi}}{\rm e}^{-\frac{(x-\mu)^2}{2\sigma^2}},
\end{equation} 
where $\sigma$ is the standard deviation and $\mu$ is the mean.

\subsection*{Exercises:}

\begin{enumerate}
\item Suppose that the lifetime of a certain tire is exponentially distributed with mean $\frac{1}{\lambda}=45,000$ miles.
\begin{enumerate}
    \marginnote{Examples of the \texttt{piecewise()} command can be found on page \pageref{sec:piecewise}.}
        \index{mathematical functions!piecewise}
	\item Use the \texttt{piecewise()} command to assign the function from equation \eqref{eq:exponential_dist} using $\lambda = \frac{1}{45000}$.
	\item Verify that this function is a valid pdf.
	\marginnote{Hint: take the integral and plot the function to verify this.}
	\item Find the probability that a given tire will last more than $40{,}000$ miles.
	\item Find the probability that a given tire will last less than $50{,}000$ miles.
	\item Find the probability that a given tire will last between $40{,}000$ and $50{,}000$ miles.
	\item Find the probability that a given tire will last exactly $40{,}000$ miles.
\end{enumerate}
\item Suppose that the height of a male is normally distributed with mean $\mu= 178$ cm and standard deviation  $\sigma= 10$ cm.
\begin{enumerate}
    \item Assign the function from equation \eqref{eq:normal_dist} using these values of $\mu$ and $\sigma$.
    \marginnote{The commands \texttt{exp()}, \texttt{sqrt()}, and \texttt{Pi} will all have to be used for this function in Maple.}
    \item Verify that this function is a valid pdf.
        \index{mathematical functions!exponential}
        \index{mathematical functions!square root}
        \index{Pi}
    \item You have a friend who is $7$ ft tall ($213$ cm). Find the probability that a given individual is that height or taller.
    \item Find the probability that a given individual is $213$ cm or smaller.
    \item What is the probability of selecting an individual with a height of exactly $213$ cm?
\end{enumerate}

\end{enumerate}