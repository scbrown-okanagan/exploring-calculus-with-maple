\section{Volumes of Revolution}\index{volume of revolution}
\label{sec:volumes_of_revolution}	

\subsection*{Recommended Tutorials:}
\begin{itemize}[noitemsep]
	\item \nameref{chp:equation_solvers}, pg. \pageref{chp:equation_solvers}
	\item \nameref{chp:implicit_functions}, pg. \pageref{chp:implicit_functions}
	\item \nameref{chp:definite_and_indefinite_Integrals}, pg. \pageref{chp:definite_and_indefinite_Integrals}
\end{itemize}

\subsection*{Introduction:}

Volumes of Revolution are often very challenging to visualize on paper. Luckily, Maple has an interactive way of visualizing the volume obtained by revolving a region about a central axis. In this activity, we will use the Volume of Revolution Tutor to find and plot the volume of a region rotated about a vertical axis or horizontal axis.

\subsection*{Exercises:}

\begin{enumerate}
    \item Define the two functions $f(x)=x^5-x^3$ and $g(x)=\sin(x)$ in Maple.
    \item 
    \begin{enumerate}
        \item Plot the graphs of $f(x)$ and $g(x)$ on the same set of axes.
            \index{plot}
        \item Find the points of intersection of the curves $f(x)$ and $g(x)$, where $x\geq0$.  You can assign this to an expression and use the expression name in the tutor.
    \end{enumerate}
    \marginnote[-1cm]{An example of finding intersection points is given on page \pageref{sec:solvingsystemeqs}.}
        \index{solving equations!intersection points}
    \item Suppose that the region between the curves $f(x)$ and $g(x)$, with $x \geq 0$, is revolved around the line $x=\pi$ to obtain a solid.
    \begin{enumerate}
        \item Using the \texttt{Int()} command, calculate the volume of revolution.
        \item Use the Volume of Revolution Tutor to plot the solid and confirm your answer in part (a).
        \marginnote[-1cm]{An example of using the Volume of Revolution Tutor can be found on page \pageref{sec:volume_of_revolution_tutor}.}
    \end{enumerate}
    \item Suppose that the region between the curves $f(x)$ and $g(x)$, with $x \geq 0$, is revolved around the line $y=-4$ to obtain a solid.
    \begin{enumerate}
        \item Use the Volume of Revolution Tutor to calculate the volume of the solid. Before closing the tutor, copy the text at the bottom (in the Maple Command box).
        \marginnote{You will need to include the \texttt{Student[Calculus1]} package by typing \texttt{with(Student[Calculus1]):} on a new line before the \texttt{VolumeOfRevolution()} command will work.}
            \index{volume of revolution!VolumeOfRevolution}
        \item Paste this command onto a new line and change \texttt{'output'=plot} to \texttt{'output'=value} to output the volume of the resulting solid.
            \index{volume of revolution!VolumeOfRevolution!output options}
    \end{enumerate}
    \clearpage
    \item Suppose you want to find the volume of an egg that has an elliptical shape defined in the $xy$-plane by $\dfrac{x^2}{2}+y^2=1$.
    \begin{enumerate}
        \marginnote{Don't forget to include the \texttt{plots} package before using \texttt{implicitplot()}.}
        \item Plot the curve using the \texttt{implicitplot()} command.
        \marginnote[1.5cm]{When plotting the ellipse, it may initially look like a circle. This is because Maple does not use the same scaling for each axis. Try clicking on the plot and using the \includegraphics[scale=0.5]{tutorials/figures/1-1.png} button in the top menu.}
            \index{packages!plots}
        \item Solve the equation of the curve for $y$ to get the equations of the top and bottom halves of the ellipse.
        \item Find the $x$-intercepts of this ellipse.
        \item Use the Volume of Revolution Tutor or the \texttt{Int()} command to calculate the area of the solid obtained by revolving the top half of the ellipse about the $x$-axis.
            \index{integral!Int}
    \end{enumerate}
\end{enumerate}