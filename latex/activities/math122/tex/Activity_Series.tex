\section{Series Convergence and Divergence}
\label{sec:series_convergence_and_divergence}	

\subsection*{Recommended Tutorials:}
\begin{itemize}[noitemsep]
    \item \nameref{chp:definite_and_indefinite_Integrals}, pg. \pageref{chp:definite_and_indefinite_Integrals}
        \index{integral}
        \index{integral!improper}
	\item \nameref{chp:sequence_and_sseries}, pg. \pageref{chp:sequence_and_sseries}
	    \index{sequences and series!}
\end{itemize}

\subsection*{Introduction:}

An infinite series is a summation of the form
\[ \sum_{n=0}^{\infty} a_n = a_0 + a_1 + a_2 + \dots. \]
For example, we can look at the geometric series
\begin{equation}
    \label{eq:geometric_series}
    \sum_{n=0}^{\infty} \frac{1}{2^n} = 1 + \frac{1}{2} + \frac{1}{4} + \frac{1}{8} + \dots.
\end{equation}
To understand the result of the sum in equation \eqref{eq:geometric_series}, we can add the first few terms, one at a time. 
\begin{align*}
    1+\frac{1}{2} &= \frac{3}{2}\\
    1+\frac{1}{2}+\frac{1}{4} &= \frac{7}{4}\\
    1+\frac{1}{2}+\frac{1}{4}+\frac{1}{8} &= \frac{15}{8}\\
\end{align*}
By looking at these partial sums, we can see that the sum is approaching the value $2$. We say that this series is \textit{convergent}. In other cases, the partial sums do not approach a finite value. We say that these series are \textit{divergent.}
    \index{sequences and series!divergent}
    \index{sequences and series!convergent}

In this activity, we will use Maple to evaluate whether a series is convergent or divergent. Maple will give the value of the sum for a convergent series and will give the value $\infty$ or $-\infty$ if the series is divergent.\index{infinity}

\subsection*{Exercises:}

\begin{enumerate}
\item For each of the following series, set up the series symbolically using the \texttt{Sum()} command. Then, use the \texttt{value(\%)} command to evaluate the sum and determine if it converges.
    \index{sequences and series!Sum}
    \index{value}
    \index{ditto operator}
\marginnote[0.5cm]{For (b), make sure you use "\texttt{Pi}" for $\pi$ and place multiplication between the $n$ and \texttt{Pi}.}
    \index{Pi}
\begin{multicols}{2}
\begin{enumerate}
    \item $\displaystyle\sum_{n=0}^{\infty} \dfrac{4^n}{n!}$
    \item $\displaystyle\sum_{n=0}^{\infty} \sin(n\pi)\arctan(n)$
    \item $\displaystyle\sum_{n=1}^{\infty} \ln(n)$
    \item $\displaystyle\sum_{n=1}^{\infty} \dfrac{n!}{n^2}$
\end{enumerate}
\end{multicols}
\end{enumerate}
\clearpage
The Integral Test for Convergence states that for a non-negative, monotonically decreasing function $f(n)$ and an integer $N$, the infinite series $$\sum_{n=N}^{\infty} f(n)$$ converges to a real number if and only if the improper integral $$\int_N^{\infty} f(x)\,dx$$ is finite. From this, we can also conclude that if the integral diverges, then the series diverges as well.\\
\begin{enumerate}
\setcounter{enumi}{1}
\item Consider the series
\[ \sum_{n=3}^{\infty}\dfrac{3}{n^2-3n+2}. \]
\begin{enumerate}
    
    \item Graph the function $f(x) = \dfrac{3}{x^2-3x+2}$ to see that the function is non-negative and monotonically decreasing over the interval $[3,\infty)$.
    \marginnote[-1.5cm]{After plotting $f(x)$, try to think about why this series starts at $n=3$.}
    \item Use the Integral Test to determine whether or not the series converges or diverges.
\end{enumerate}
\end{enumerate}