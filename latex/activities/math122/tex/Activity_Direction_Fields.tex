\section{Direction Fields}
\label{sec:direction_fields}	

\subsection*{Recommended Tutorials:}
\begin{itemize}[noitemsep]
	\item \nameref{chp:differential_equations}, pg. \pageref{chp:differential_equations}
	    \index{differential equations}
\end{itemize}

\subsection*{Introduction:}
In this activity, you will use direction fields to predict the population dynamics for a population of rabbits. Suppose that you have a population of rabbits and $P(t)$ is the number of rabbits at time $t$. 

\marginnote{You will need to include the \texttt{DETools} package to use the \texttt{DEplot()} command.}
    \index{differential equations!DETools}
    \index{packages!differential equations!DETools!DEplot()}

We can model the population of rabbits using a basic model
\begin{equation}
\label{eq:rabbits_basic}
\frac{dP}{dt}=\alpha P - \beta P,
\end{equation}
where $\alpha$ is the birth rate and $\beta$ is the death rate.

Equation \eqref{eq:rabbits_basic} does not consider limitations due to habitat and food supply. If we wish to use a more accurate model, then we can consider the logistic growth model
\begin{equation}
\label{eq:rabbits_logistic}
\frac{dP}{dt}= kP \left(1-\frac{P}{M}\right),
\end{equation}
where $k$ is the relative growth rate and $M$ is the carrying capacity (the maximum population that is sustainable). 

The death rate of the rabbits (due to hunting) can be added to this logistic model to obtain the differential equation
\begin{equation}
\label{eq:rabbits_logistic_hunted}
\frac{dP}{dt}=kP \left(1-\frac{P}{M}\right)-bP,
\end{equation}
where $b$ is the hunting rate. 

We will examine solutions to equations \eqref{eq:rabbits_basic}--\eqref{eq:rabbits_logistic_hunted} using two different initial conditions: $P(0)=1$ and $P(0)=50$.

\subsection*{Exercises:}

\marginnote[0.5cm]{Remember to use $P(t)$ and not just $P$ in your differential equation.}
\begin{enumerate}
\item Consider the basic population model in equation \eqref{eq:rabbits_basic}.
\begin{enumerate}
    \item Draw the direction field using $\alpha=2$ and $\beta=1$, including both initial conditions given.
    \item Draw the direction field using $\alpha=1$ and $\beta=2$, including both initial conditions given.
    \item In a new paragraph in your worksheet, describe what you can conclude about the importance of the death to birth rate comparison.
\end{enumerate}
\clearpage
\item Consider the logistic growth model in equation \eqref{eq:rabbits_logistic}.
\label{ex:rabbits_logistic}
\begin{enumerate}
    \item Draw the direction field using $k=2$ and $M=30$, including both of the initial conditions given.
        \index{differential equations!DEplot}
    \item In a new paragraph, describe what you observe about the solutions.
\end{enumerate}
\item Consider the logistic growth model in equation \eqref{eq:rabbits_logistic_hunted}.
\begin{enumerate}
    \item Draw the direction field using $k=2$, $M=30$, and $b=1$, including both of the initial conditions given.
    \item In a new paragraph in your worksheet, describe what changed from exercise \ref{ex:rabbits_logistic} by adding the death rate to the differential equation.
\end{enumerate} 
\end{enumerate}

Most mammal population growth is dependent upon other species in the region, via an interconnected food web. One simple predator-prey model is the Lotka-Volterra model
\begin{align}
\label{eq:rabbits_lotka_volterra}
\frac{dx}{dt} &= \alpha x-\beta xy\\
\frac{dy}{dt} &= \gamma xy-\delta y,\nonumber
\end{align}
\marginnote[-1.25cm]{Make sure you insert multiplication between $x(t)$ and $y(t)$ here, otherwise Maple will think you want to use a variable called $xy$.}
where $x(t)$ is the population of prey and $y(t)$ is the population of predators. In this equation, the prey grow and are eaten by predators. The predators' growth depends on eating the prey and the predators have a death rate.

\begin{enumerate}
\setcounter{enumi}{3}
\item Consider the Lotka-Volterra model in equation \eqref{eq:rabbits_lotka_volterra}.
\begin{enumerate}
    \item Go to Tools, Tutors, Differential Equations, DE Plots, select Lotka-Volterra Model, and then DEPlot. If you want, you can change the parameters or the initial conditions. Then click quit to plot it on your Maple worksheet. 
    \item In a new paragraph, explain what the prey and predator populations do on this direction field. Notice that the prey is on the $x$-axis and the predator is on the $y$-axis of the direction field.
    \index{differential equations!DETools}
    \index{packages!differential equations!DETools!DEplot()}
\end{enumerate}
\end{enumerate}
