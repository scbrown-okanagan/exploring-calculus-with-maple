\chapter{List of Common Commands}
\label{chp:list_of_common_commands}

\begin{fullwidth}
%\begin{landscape}
% Table generated by Excel2LaTeX
\begin{longtable}{p{0.33\linewidth} p{0.66\linewidth}}
    \multicolumn{2}{l}{\textbf{Keyboard Shortcuts}} \\
    \hline
    Ctrl+J, Ctrl+K & New execution group (after or before current line) \\
    Ctrl+Shift+J, Ctrl+Shift+K & New  paragraph (after or before current line) \\
    Ctrl+T & Convert line to Paragraph \\
    Ctrl+M & Convert line to Maple input \\
    Ctrl+. & Indent section \\
    Ctrl+, & Unindent section \\
    Ctrl+Delete & Delete section \\
    F4 & Merge consecutive execution groups \\
    F5 & Toggle between Text, Nonexecutable Math, and Math types \\
          &  \\
    \multicolumn{2}{l}{\textbf{Common Expressions}} \\
    \hline
    exp(x) & The exponential function \\
    sqrt(x) & The square root function \\
    abs(x) & The absolute value function \\
    surd(x,n) & The primitive $n$\textsuperscript{th} root, $x^{1/n}$ \\
          &  \\
    \multicolumn{2}{l}{\textbf{Manipulating Expressions}} \\
    \hline
    name := expression & Assigns an expression to a variable \\
    subs(x=a, expression) & Evaluate an expression at $x = a$ \\
    evalf(expression) & Evaluate the given expression as a (floating point) decimal \\
    factor(expression) & Factor the given expression \\
    simplify(expression) & Simplify the given expression \\
    expand(expression) & Expand the given expression \\
    collect(expression,var) & Collect terms of the expression by the specified variable \\
          &  \\
    \multicolumn{2}{l}{\textbf{Solving Equations}} \\
    \hline
    solve(equation,var) & Solves the given equation for the specified variable \\
    fsolve(equation,var) & Solves the given equation for the specified variable (as a decimal) \\
    fsolve(equation,var=a..b) & Solves the given equation for the specified variable (as a decimal) on the interval $[a,b]$ \\
    solve(\{eqn1,eqn2\},\{var1,var2\}) & Solves a system of equations for all specified variables \\
    fsolve(\{eqn1,eqn2\},\{var1,var2\}) & Solves a system of equations for all specified variables (as a decimal) \\
          &  \\
    \multicolumn{2}{l}{\textbf{Defining Functions}} \\
    \hline
    name(var) := expression & Assigns a function of the specified variable \\
    name := unapply(expression,var) & Convert an expression to a function of the specified variable\\
    name(var) := piecewise(condition,
 & Create a piecewise function of the specified variable where\\
    \hspace{1.5cm}   expr,…,condition,expr,…) & each condition is an interval\\
          &  \\
    \multicolumn{2}{l}{\textbf{Plotting Functions}} \\
    \hline
    plot(f(x),x=a..b) & Plot the given function, $f(x)$, over the interval $[a,b]$ \\
    plot([f(x),g(x)],x=a..b) & Plot two functions, $f(x)$ and $g(x)$, over the interval $[a,b]$ \\
          &  \\
    \multicolumn{2}{l}{\quad\quad\textbf{Additional Plot Parameters} (Include these after necessary parameters)} \\
    \quad\quad y=c..d & Only plot the range $c \leq y \leq d$ \\
   \quad\quad discont=true & Includes discontinuities in a plot \\
    \quad\quad colour=blue & Specify the colour for a graph (black, blue, red, etc.) \\
    \quad\quad linestyle=dotted & Specify the style of the line (dash, dot, etc.) \\
          &  \\
    \multicolumn{2}{l}{\textbf{Limits}} \\
    \hline
    limit(expression,var=a) & Find the limit of the expression as var approaches $a$ \\
    limit(expression,var=a,right) & Find the limit of the expression as var approaches $a$ from the right \\
    limit(expression,var=a,left) & Find the limit of the expression as var approaches $a$ from the left \\
    limit(expression,var=infinity) & Find the limit of the expression as var approaches infinity \\
          &  \\
    \multicolumn{2}{l}{\textbf{Derivatives}} \\
    \hline
    diff(expression,var) & The derivative of the given expression with respect to variable \\
    diff(expression,var,var) & The second derivative of the given expression with respect to variable \\
    diff(expression,var\$2) & The second derivative of the given expression with respect to variable \\
    f'(var) & The derivative of the function $f$ with respect to variable\\
    f\symbol{94}(n)(var) & The nth derivative of the function $f$ with respect to variable\\
          &  \\
    \multicolumn{2}{l}{\textbf{Implicit Functions} (requires "plots" package for plotting)} \\
    \hline
    implicitplot(equation,x=a..b,y=c..d) & Plot the implicit function over the specified region \\
    implicitdiff(equation,y,x) & The derivative of the implicit function, given as $dy/dx$ \\
          &  \\
        \newpage
    \multicolumn{2}{l}{\textbf{Riemann Sums and Numerical Integration} (requires "Student[Calculus1]" package)} \\
    \hline
    ApproximateInt(f(x),x=a..b) & Approximate the definite integral of $f(x)$ from $x=a$ to $x=b$ \\
          &  \\
    \multicolumn{2}{l}{\quad\quad\textbf{Additional ApproximateInt Parameters} (Include these after necessary parameters)} \\
    \quad\quad method=left & Choose left rectangles for approximation \\
    \quad\quad method=right & Choose right rectangles for approximation \\ 
    \quad\quad method=lower & Choose lower bound rectangles for approximation \\
    \quad\quad method=upper & Choose upper bound rectangles for approximation \\
    \quad\quad method=midpoint & Choose midpoint rectangles for approximation \\ 
    \quad\quad method=trapezoid & Choose trapezoid rule approximation \\
    \quad\quad method=simpson & Choose Simpson's rule approximation \\
    \quad\quad output=sum & Output summation notation for given approximation method \\
    \quad\quad output=value & Output exact value of approximation \\
    \quad\quad output=plot & Output graph of integrand function and approximation \\
    \quad\quad output=animation & Output animation of approximation as $n$ increases \\
    \quad\quad partition=n & Use $n$ equally spaced subintervals for approximation \\
          &  \\
    \multicolumn{2}{l}{\textbf{Sequences and Series}} \\
    \hline
    expression \$ var=a..b & Display the sequence of the expression from var = $a$ up to var = $b$ \\
    seq(expression,var=a..b) & Display the sequence of the expression from var = $a$ up to var = $b$ \\
    Sum(expression,var=a..b) & Display the sum for the expression from var = $a$ up to var = $b$) \\
    sum(expression,var=a..b) & Give the value of the sum of the expression from var = $a$ up to var = $b$ \\
    taylor(f(x),x=a,n) & Give the Taylor series expansion of $f(x)$ about $x=a$, including terms up to power $n-1$ \\
          &  \\
    \multicolumn{2}{l}{\textbf{Integrals}} \\
    \hline
    Int(f(x),x) & The indefinite integral, display "inert" form \\
    int(f(x),x) & The indefinite integral, evaluated \\
    Int(f(x),x=a..b) & The definite integral over the specified interval, display "inert" form \\
    int(f(x),x=a..b) & The definite integral over the specified interval, evaluated \\
          &  \\
    \multicolumn{2}{l}{\textbf{Differential Equations}} \\
    \hline
    dsolve(DE, y(x)) & Solves the given differential equation for $y(x)$ \\
    dsolve([DE, ICs], y(x)) & Solves the given differential equation with initial conditions for $y(x)$ \\
          &  \\
    \multicolumn{2}{l}{\textbf{Direction Fields} (Requires “DEtools” package)} \\
    \hline
    DEplot(DE,y(x),x=a..b,y=a..b) & Plot the direction field for the differential equation $dy/dx$  \\
    DEplot(DE,y(x),x=a..b,y=a..b,[ICs]) & Plot the direction field for the differential equation $dy/dx$ with initial conditions \\
          &  \\
    \multicolumn{2}{l}{\quad\quad\textbf{Additional DEplot Parameters} (Include these after necessary parameters)} \\
    \quad\quad arrows=line & Use lines for the direction field, rather than arrows \\
    \quad\quad colour=black & Change arrow colour \\
    \quad\quad linecolour=blue & Change solution curve colour \\
  \label{commands}%
\end{longtable}%
%\end{landscape}
\end{fullwidth}